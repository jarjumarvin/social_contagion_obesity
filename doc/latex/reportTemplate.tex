\documentclass[11pt]{article}
\usepackage{geometry}
\geometry{letterpaper}

\usepackage{graphicx}
\usepackage{amssymb}
\usepackage{epstopdf}
\usepackage{natbib}
\usepackage{amssymb, amsmath}
\DeclareGraphicsRule{.tif}{png}{.png}{`convert #1 `dirname #1`/`basename #1 .tif`.png}

%\title{Title}
%\author{Name 1, Name 2}
%\date{date}

\begin{document}



\thispagestyle{empty}

\begin{center}
\includegraphics[width=5cm]{images/ETHlogo.eps}

\bigskip


\bigskip


\bigskip


\LARGE{ 	Agent-Based Modeling:\\ }
\LARGE{ Social System Simulation\\}

\bigskip

\bigskip

\small{Project Report}\\

\bigskip

\bigskip

\bigskip

\bigskip


\begin{tabular}{|c|}
\hline
\\
\textbf{\LARGE{Spread of Obesity in Social Networks}}\\
\\
\hline
\end{tabular}
\bigskip

\bigskip

\bigskip

\LARGE{Chio Ge, Marvin Jarju \& Giulia Zheng}



\bigskip

\bigskip

\bigskip

\bigskip

\bigskip

\bigskip

\bigskip

\bigskip

Zurich\\
December 2019\\

\end{center}



\newpage

%%%%%%%%%%%%%%%%%%%%%%%%%%%%%%%%%%%%%%%%%%%%%%%%%

\newpage
\section*{Agreement for free-download}
\bigskip


\bigskip


\large We hereby agree to make our source code for this project freely available for download from the web pages of the SOMS chair. Furthermore, we assure that all source code is written by ourselves and is not violating any copyright restrictions.

\begin{center}

\bigskip


\bigskip


\begin{tabular}{@{}p{3.3cm}@{}p{6cm}@{}@{}p{6cm}@{}}
\begin{minipage}{3cm}

\end{minipage}
&
\begin{minipage}{6cm}
\vspace{2mm} \large Name 1

 \vspace{\baselineskip}

\end{minipage}
&
\begin{minipage}{6cm}

\large Name 2

\end{minipage}
\end{tabular}


\end{center}
\newpage

%%%%%%%%%%%%%%%%%%%%%%%%%%%%%%%%%%%%%%%



% IMPORTANT
% you MUST include the ETH declaration of originality here; it is available for download on the course website or at http://www.ethz.ch/faculty/exams/plagiarism/index_EN; it can be printed as pdf and should be filled out in handwriting


%%%%%%%%%% Table of content %%%%%%%%%%%%%%%%%

\tableofcontents

\newpage

%%%%%%%%%%%%%%%%%%%%%%%%%%%%%%%%%%%%%%%



\section{Abstract}

\section{Individual contributions}

\section{Introduction and Motivations}
Human behaviour is chiefly governed by societal and environmental influences. Social interactions, nurture and upbringing account for an individual’s identity, their choices and way of life, which gives rise to the notion that sociological and behavioural phenomena could be transmitted via social networks in a matter not unlike a viral contagion.\\

Traditionally, the term "contagion" is associated with strictly epidemiological factors, i.e. infectious diseases or viruses that spread through networks by human contact. In a similar vein, we can extend this connotation by considering a "social contagion" as a behavioural pattern that propagates interpersonally through a \textit{social} network via interaction. Thus, social contagions have the ability to become quintessential determinants for an individual’s health and mental well-being. Based on this premise, it enables us to study behavioural phenomena using network analytics and classic epidemiological models, albeit most of those will be subject to slight modifications in order to be compatible with the problems we are concerned with.\\

One such instance of a recent issue we are facing is the prevalence of obesity in modern society, which may be regarded as an epidemic in its own right. As of 2017, approximately 40\% of the Swiss populated is overweight and a further 10\% is obese. In juxtaposition to figures of 1992, this represents a full 100\% increase in the last 25 years. This issue is not limited to Switzerland, by all means – We observe similar trends, if not starker ones, throughout the rest of the globe (in fact, Switzerland holds up comparatively well in that regard). Hitherto, the obesity epidemic cannot be adequately explained by genetic factors (genetics may merely affect one’s predisposition towards obesity) and the epidemic permeates through all ages and socioeconomic groups. Hence, as we are lacking congenital or medicals explanations to expound this sudden upsurge, this allows our theory of a social contagion to step to the fore.\\

We hypothesise that an individual’s attitude towards obesity and one’s habits change as those around them do. For instance, a person’s tolerance for weight-gain may be increased if they converse with multiple contacts that are also obese and group activities like eating-out or smoking may be forced upon an individual by means of peer pressure. Furthermore, if a child has obese parents, this might reflect in the child’s upbringing and diet.\\

In the following, we devise a model to simulate these aforementioned considerations on the basis of a social network that will be representative of the demographics of Switzerland. We will try interpolate between the figures we observe from 1992 and 2017 and moreover, attempt to make a prediction for the “disease” process in the years to come. Ideally, we will want to confirm the correlation between the spread of obesity and social influence, or in the very least, showcase social contagions and the modelling of those as a viable consideration for a plethora of behavioural phenomena and pseudo-epidemics.


\section{Description of the Model}

\section{Implementation}

\section{Simulation Results and Discussion}

\section{Summary and Outlook}

\section{References}






\end{document}
